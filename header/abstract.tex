%
% abstract.tex
%
% Copyright (C) 2022 by Universidade Federal de Santa Catarina.
%
% GNSS Networks Based on Small Satellites
%
% This work is licensed under the Creative Commons Attribution-ShareAlike 4.0
% International License. To view a copy of this license,
% visit http://creativecommons.org/licenses/by-sa/4.0/.
%

%
% \brief Abstract page.
%
% \author Gabriel Mariano Marcelino <gabriel.mm8@gmail.com>
%
% \version 0.1.0
%
% \date 2022/06/24
%

\chapter*{Abstract}

{\parindent0pt
%Redes de navegação e posicionamento utilizando constelações de satélites já são utilizadas a décadas tanto em escala global quanto regional. Mas até então, somente eram empregados satélites de grande porte operando em órbitas médias. Com o advento do satélites de pequeno porte como os CubeSats, novas estratégias de desenvolvimento e implementação vêm surgindo, visando principalmente simplificar e reduzir o custo de implantação de novos sistemas ou teste de novas tecnologias em órbita. Desta forma, este trabalho propõe um estudo da viabilidade do emprego de uma rede de posicionamento utilizando satélites de pequeno porte, visando principalmente redução de custo e simplificação da implantação deste tipo de sistema. Como forma de validar o conceito, propõe-se o desenvolvimento de uma carga útil que contendo um transmissor de sinais de GNSS, além da definição de uma missão completa para a validação do sistema e do conceito, através uma constelação.
Navigation and positioning networks using satellite constellations have been used for decades on both a global and regional scale. However, until now, only large satellites operating in medium orbits have been used. With the advent of small satellites such as CubeSats, new strategies for development and implementation are emerging, aiming mainly to simplify and reduce the cost of deploying new systems or testing new technologies in orbit. Thus, this work proposes a feasibility study of the use of a positioning network using small satellites, aiming mainly to reduce costs and simplify the deployment of this type of system. As a way of validating the concept, the development of a payload containing a GNSS signal transmitter is proposed, as well as the definition of a complete mission for the validation of the system and concept, through a constellation.
}

\smallskip
\noindent \textbf{Keywords:} Embedded systems. Satellites. GNSS. Constellations.
