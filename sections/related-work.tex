%
% related-work.tex
%
% Copyright (C) 2021 by Universidade Federal de Santa Catarina.
%
% On-Board Data Processing Techniques to Improve the Performance of Small Satellites
%
% This work is licensed under the Creative Commons Attribution-ShareAlike 4.0
% International License. To view a copy of this license,
% visit http://creativecommons.org/licenses/by-sa/4.0/.
%

%
% \brief Related works section.
%
% \author Gabriel Mariano Marcelino <gabriel.mm8@gmail.com>
%
% \version 0.0.0
%
% \date 2021/06/14
%

\section{Related Work} \label{sec:related-work}

\cite{cosmas2020}

\cite{mateo-garcia2021}

Phi-Sat-1 (ou $\Phi$-Sat-1) \cite{phi-sat-1} é um CubeSat 6U e é considerado o primeiro satélite europeu com uma inteligência artifical embarcada. O objetivo é analisar imagens capturadas por uma câmera embarcada e descardas imagens com uma predominância de nuvens, ou seja, descartar já em voo fotos sem interesse para a missão.

SpaceX usa Linux em todos os projetos, não usa components rad-hard e aposta na redundância. Linux com kernel modificado \cite{leppinen2017}.

Planet utiliza Linux embarcado para fazer pré-processamento e economizar banda (uso intensivo de banda para envio de imagens em tempo real) \cite{leppinen2017}.
