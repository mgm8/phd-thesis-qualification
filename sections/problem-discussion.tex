%
% problem-discussion.tex
%
% Copyright (C) 2021 by Universidade Federal de Santa Catarina.
%
% On-Board Data Processing Techniques to Improve the Performance of Small Satellites
%
% This work is licensed under the Creative Commons Attribution-ShareAlike 4.0
% International License. To view a copy of this license,
% visit http://creativecommons.org/licenses/by-sa/4.0/.
%

%
% \brief Problem discussion section.
%
% \author Gabriel Mariano Marcelino <gabriel.mm8@gmail.com>
%
% \version 0.1.0
%
% \date 2021/06/14
%

\section{Problem Discussion} \label{sec:problem-discussion}

\cite{marcelino2018}

\cite{marcelino2016}

\begin{itemize}
    \item IA embarcada
    \item Voo em formação
    \item Constelação de nanossatélites
    \item Processamento distribuído (constelação)
\end{itemize}

\subsection{Processing Hardware}

\begin{itemize}
    \item Nvidia Jetson
    \item Intel Movidius
    \item FPGA
\end{itemize}

Nvidia Jetson is a series of embedded computing boards from Nvidia, based on a Tegra SoC that integrates an ARM CPU. Jetson is a low-power system and is designed for accelerating machine learning applications.

\subsection{Descrição de hardware}

Guppy \cite{al-dujaili2012} (GPU-like cUstomizable Parallel Processor prototYpe) is based on an existing general purpose parameterizable soft core, namely the LEON3.

RISC-V \cite{waterman2016}

The NOEL-V \cite{andersson2020} is a synthesizable VHDL model of a processor that implements the RISC-V architecture. The processor is the first released model in our RISC-V line of processors that complement the LEON line of processors.

The NOEL-V can be implemented as a dual-issue processor, allowing up to two instructions per cycle to be executed in parallel. To support the instruction issue rate of the pipeline, the NOEL-V has advanced branch prediction capabilities. The cache controller of the NOEL-V supports a store buffer FIFO with one cycle per store sustained throughput, and wide AHB slave support to enable fast stores and fast cache refill.

The NOEL-V is interfaced using the AMBA 2.0 AHB bus (subsystem with Level-2 cache and AXI4 backend is also available) and supports the IP core plug\&play method provided in the our IP library (GRLIB). The processor can be efficiently implemented on FPGA and ASIC technologies and uses standard synchronous memory cells for caches and register file.
