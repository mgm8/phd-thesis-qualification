%
% problem-discussion.tex
%
% Copyright (C) 2022 by Universidade Federal de Santa Catarina.
%
% GNSS Networks Based on Small Satellites
%
% This work is licensed under the Creative Commons Attribution-ShareAlike 4.0
% International License. To view a copy of this license,
% visit http://creativecommons.org/licenses/by-sa/4.0/.
%

%
% \brief Problem discussion section.
%
% \author Gabriel Mariano Marcelino <gabriel.mm8@gmail.com>
%
% \version 0.1.0
%
% \date 2021/06/14
%

\section{Problem Discussion} \label{sec:problem-discussion}

Os sistemas de navegação global por satélite atualmente são largamente utilizados por diversos dispositivos, incluindo desde sistemas complexos como aeronaves, satélites e foguetes, até disponíveis mais simples de uso geral como celulares, computadores, rastreadores de veículos, etc.

Atualmente existem N diferentes redes em operação, sendo um tipo de tecnologia dominada por poucos países do mundo. Todas as redes atuais operam com a utilização de satélites de grande porte e operando em órbitas baixas e médias (LEO e MEO).

Com o surgimento e ascenção dos pequenos satélites, em especial os CubeSats, na última década, surge também a possibilidade de utilizar satélites de menor porte, mais simples e de menor custo para aplicações que anteriormente só poderiam ser solucionadas com satélites de grande porte e custo elevado.

Especificamente sobre as redes de GNSS, o uso de satélites de pequeno porte pode agregar inúmeras vantagens, sendo umas das principais a redução de custo (tanto de desenvolvimento e operação), e 
