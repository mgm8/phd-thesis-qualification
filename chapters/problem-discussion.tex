%
% problem-discussion.tex
%
% Copyright (C) 2022 by Universidade Federal de Santa Catarina.
%
% GNSS Networks Based on Small Satellites
%
% This work is licensed under the Creative Commons Attribution-ShareAlike 4.0
% International License. To view a copy of this license,
% visit http://creativecommons.org/licenses/by-sa/4.0/.
%

%
% \brief Problem discussion chapter.
%
% \author Gabriel Mariano Marcelino <gabriel.mm8@gmail.com>
%
% \version 0.1.0
%
% \date 2021/06/14
%


\chapter{Problem Discussion} \label{ch:problem-discussion}

Os sistemas de navegação global por satélite atualmente são largamente utilizados por diversos dispositivos, incluindo desde sistemas complexos como aeronaves, satélites e foguetes, até disponíveis mais simples de uso geral como celulares, computadores, rastreadores de veículos, etc.

Atualmente existem N diferentes redes em operação, sendo um tipo de tecnologia dominada por poucos países do mundo. Todas as redes atuais operam com a utilização de satélites de grande porte e operando em órbitas baixas e médias (LEO e MEO).

Com o surgimento e ascenção dos pequenos satélites, em especial os CubeSats, na última década, surge também a possibilidade de utilizar satélites de menor porte, mais simples e de menor custo para aplicações que anteriormente só poderiam ser solucionadas com satélites de grande porte e custo elevado.

Especificamente sobre as redes de GNSS, o uso de satélites de pequeno porte pode agregar inúmeras vantagens, sendo umas das principais a redução de custo (tanto de desenvolvimento e operação), e 

\subsection{Timing precision}

% https://gssc.esa.int/navipedia/index.php/Precise_Time_Reference#cite_note-About-2

GNSS technologies have a design dependence on accurate timing. The resolution of positioning equations depend on the accurate timestamping of GNSS messages and the four variables resolved by positioning equations are: time plus the 3D position coordinates. Each navigation satellite has atomic clocks that are synchronized from a master clock on the ground and the navigation messages are timestamped with the transmission time of the signal.

This allows GNSS receivers to act as a worldwide synchronized time source with a precision that could only be maintained during long periods by expensive equipments. This enabled a wide set of applications that rely on GNSS synchronized precise time sources. These applications can range from network synchronization and optimization to encryption and digital signature of electronic data.

The receiver's clocks, however, are small quartz oscillators like those found in a wristwatch. Quartz oscillators are very accurate when measuring times of less than a few seconds, but rather inaccurate over longer periods. The solution is to re-set the receiver’s time to the satellite’s time continuously. This is done by the receiver’s processor using an approximation method involving signals from at least four satellites \cite{about-satellite-nagivation}.

The accuracy achieved by GNSS-based time synchronization using GPS is <40 ns 95 \% of time \cite{gps-standard}.

\cite{sa45s}

\begin{figure}[!ht]
    \begin{center}
        \includegraphics[width=0.7\columnwidth]{figures/microchip-csac}
        \caption{Microchip CSAC SA.45S atomic clock.}
        \label{fig:microchip-csac}
    \end{center}
\end{figure}

\subsection{Telecommunication analysis}

\subsubsection{Distance to Satellite at Horizon}

The distance to satellite at horizon (the maximum theoretical distance between the satellite and a ground station) can be calculated using \autoref{eq:horizon-distance}.

\begin{equation} \label{eq:horizon-distance}
d = \sqrt{2\cdot R_{e}\cdot h + h^{2}}
\end{equation}

Where:

\begin{itemize}
    \item $R_{e}$ = Earth radius = 6378 km
    \item $h$ = Satellite altitude = 550 km
    \item $d$ = Distance to sattellite at horizon
\end{itemize}

So, the distance to satellite at horizon is:

\begin{equation} \label{eq:horizon-distance-result}
d = \sqrt{2\cdot 6378\cdot 550 + 550^{2}} = \mathbf{2705\ km}
\end{equation}

\subsubsection{Free-Space Path Loss}

The free-space path loss ($FSPL$\nomenclature{\textbf{FSPL}}{Free-Space Path Loss}) can be calculated using \autoref{eq:fspl}.

\begin{equation} \label{eq:fspl}
FSPL = \left( \frac{4\pi d f}{c} \right)^{2}
\end{equation}

Where:

\begin{itemize}
    \item $d$ = Distance between the satellite and the ground station
    \item $f$ = Radio frequency
    \item $c$ = Speed of light
\end{itemize}

The FSPL value in decibels can be calculated with \autoref{eq:fsbl-db}.

\begin{equation} \label{eq:fsbl-db}
    \begin{split}
        FSPL^{dB} & = 20\log\left(\frac{4\pi}{c}\right) + 20\log\left(d\right) + 20\log\left(f\right) \\
                  & = 32,45 + 20\log\left(\frac{d}{1\ km}\right) + 20\log\left(\frac{f}{1\ MHz}\right) \\
    \end{split}
\end{equation}

The minimum distance between the satellite and a ground station is the satellite altitude, in this case: 600 km. The maximum distance is the distance at horizon, defined by \autoref{eq:horizon-distance-result}.

Considering the frequency of the beacon as 437 MHz, the minimum and maximum FSBL is:

\begin{equation}
    FSPL^{dB}_{min} = 32,45 + 20\log\left(\frac{550}{1\ km}\right) + 20\log\left(\frac{146}{1\ MHz}\right) = \mathbf{130,5\ dB}
\end{equation}

\begin{equation}
    FSPL^{dB}_{max} = 32,45 + 20\log\left(\frac{2705}{1\ km}\right) + 20\log\left(\frac{146}{1\ MHz}\right) = \mathbf{144,4\ dB}
\end{equation}

\begin{equation}
    \mathbf{130,5 \leq FSPL^{dB} \leq 144,4\ dB}
\end{equation}

\subsubsection{Power at Receiver}

The power of the signal at the receiver can be estimated using \autoref{eq:power-at-receiver}.

\begin{equation} \label{eq:power-at-receiver}
    P_{r} = P_{t} + G_{t} + G_{r} - L_{p} - L_{s}
\end{equation}

Where:

\begin{itemize}
    \item $P_{r}$ = Power at the receiver
    \item $P_{t}$ = Transmitter power
    \item $G_{t}$ = Antenna gain of the transmitter
    \item $G_{r}$ = Antenna gain of the receiver
    \item $L_{p}$ = FSPL (Free-Space Path Loss)
    \item $L_{s}$ = Other losses in the system
\end{itemize}

Considering the worst scenario with the maximum possible distance between the satellite and a ground station, the power at the receiver for the link is calculated below.

\begin{equation}
    P_{r} = 30 + 0 + 12 - 144,4 - 5 = -107,4\ dBm
\end{equation}

\begin{equation}
    \mathbf{P_{r} \geq -107,4\ dBm}
\end{equation}

\subsubsection{Signal-to-Noise-Ratio}

The Signal-to-Noise-Ratio (SNR\nomenclature{\textbf{SNR}}{Signal-to-Noise-Ratio}) of a transmitted signal at the receiver can be expressed using \autoref{eq:snr}:

\begin{equation} \label{eq:snr}
    SNR = \frac{E_{b}}{N_{0}} = \frac{P_{t}G_{t}G_{r}}{kT_{s}RL_{p}}
\end{equation}

Where:

\begin{itemize}
    \item $P_{t}$ = Transmitter power
    \item $G_{t}$ = Antenna gain of the transmitter
    \item $G_{r}$ = Receiver gain
    \item $k$ = Boltzmann's constant ($\cong 1,3806 \times 10^{-23}\ J/K$)
    \item $T_{s}$ = System noise temperature
    \item $R$ = Data rate in bits per seconds (bps)
    \item $L_{p}$ = Free-Space Path Loss (FSPL)
\end{itemize}

The system noise temperature ($T_{s}$) can be defined using \autoref{eq:system-noise-temperature}.

\begin{equation} \label{eq:system-noise-temperature}
    T_{s} = T_{ant} + T_{r}
\end{equation}

with:

\begin{equation} \label{eq:noise-temperature-receiver}
    T_{r} = \frac{T_{0}}{L_{r}} (F - L_{r})
\end{equation}

and:

\begin{equation} \label{eq:noise-figure}
    F = 1 + \frac{T_{r}}{T_{0}}
\end{equation}

Combining Equations \ref{eq:system-noise-temperature}, \ref{eq:noise-temperature-receiver} and \ref{eq:noise-figure}:

\begin{equation} \label{eq:system-noise-temp-expanded}
    T_{s} = T_{ant} + \left( \frac{T_{0}(1 - L_{r})}{L_{r}} \right) + \left( \frac{T_{0} (F - 1)}{L_{r}} \right)
\end{equation}

Where:

\begin{itemize}
    \item $T_{ant}$ = Antenna noise temperature
    \item $T_{0}$ = Reference temperature (usually 290 K)
    \item $L_{r}$ = Line loss between the antenna and the receiver
    \item $F$ = Noise figure of the receiver
    \item $T_{r}$ = Noise temperature of the receiver
\end{itemize}

The SNR value in decibels can be calculated using the \autoref{eq:snr-db}:

\begin{equation} \label{eq:snr-db}
    \begin{split}
        SNR^{dB} & = 10\log_{10}\left( \frac{E_{b}}{N_{0}} \right) = 10\log_{10} \left( \frac{P_{t}G_{t}G_{r}}{kT_{s}RL_{p}} \right) \\
                 & = P_{t}^{dBm} - 30 + G_{t}^{dB} + G_{r}^{dB} - L_{p}^{dB} - 10\log k - 10\log T_{s} - 10\log R
    \end{split}
\end{equation}

Considering other losses in the system ($L_{s}$) (cable and connection losses as example), the \autoref{eq:snr-db} can be corrected as presented in \autoref{eq:snr-db-with-losses}.

\begin{equation} \label{eq:snr-db-with-losses}
    SNR^{dB} = P_{t}^{dBm} - 30 + G_{t}^{dB} + G_{r}^{dB} - L_{p}^{dB} - L_{s}^{dB} - 10\log k - 10\log T_{s} - 10\log R
\end{equation}

Using Equations \ref{eq:snr-db-with-losses} and \ref{eq:system-noise-temperature}, with:

\begin{itemize}
    \item $P_{t} = 30\ dBm$
    \item $G_{t} = 0\ dBi$
    \item $G_{r} = 12\ dBi$
    \item $L_{p} = 144,4\ dB$
    \item $L_{s} = 5\ dB$
    \item $R = 1200\ bps$
    \item $T_{0} = 290\ K$
    \item $T_{r} = 290\ K$
    \item $T_{ant} = 300\ K$
    \item $F = 2\ dB$
    \item $L_{r} = 0,89\ (0,5\ dB)$
\end{itemize}

\begin{equation}
    T_{s} = 300 + \left( \frac{290 (1 - 0,89)}{0,89} \right) + \left( \frac{290 (2 - 1)}{0,89} \right) = 661,7\ K
\end{equation}

\begin{equation}
    SNR^{dB} = 30 - 30 + 0 + 12 - 144,4 - 5 + 228,6 - 28,21 - 30,79 = 32,22\ dB
\end{equation}

\begin{equation}
\mathbf{SNR^{dB} \geq 32,22\ dB}
\end{equation}

\subsubsection{Link Margin}

From \cite{larson2005}, the minimum SNR value at the received considering a $10^{-5}$ bit error rate is:

\begin{itemize}
    \item Beacon: $SNR^{dB} \geq 9,6\ dB$
\end{itemize}

And considering the link margin as the SNR of the link minus the SNR threshold for a given bit error, the link margin of the radio links of the satellite are:

\begin{itemize}
    \item Beacon: $32,22 - 9,6 = \mathbf{22,62\ dB}$
\end{itemize}

\subsection{Power consumption analysis}

According to section 10.3 of \cite{larson2005}, the power budget of satellite can be determined through three steps:

\begin{enumerate}
    \item Prepare operating power budget
    \item Size the battery
    \item Estimate power degradation over mission life
\end{enumerate}

\subsubsection{Input Power}

A simulation of the energy input to the solar panels along some orbits can be seen in the \autoref{fig:sp_sim_power} graph. From this simulation, the following results were obtained:

\begin{itemize}
    \item Peak power $\cong$ 8759,5 mW
    \item Average (orbit) $\cong$ 2744,9 mW
    \item Average (sunlight) $\cong$ 4315,6 mW
    \item Orbit period $\cong$ 6018 sec
    \item Sun light period $\cong$ 3712 sec
    \item Eclipse period $\cong$ 2124 sec
\end{itemize}

\begin{figure}[!ht]
    \begin{center}
        \includegraphics[width=\columnwidth]{curves/sp_sim_power}
        \caption{Silumated input power of the solar panels.}
        \label{fig:sp_sim_power}
    \end{center}
\end{figure}

\subsubsection{Operating Power Budget}

Typical operating voltages, and current and power ranges consumed by each satellite subsystem are presented in \autoref{tab:power-requirements}.

\begin{table}[!h]
    \centering
    \begin{tabular}{lccccc}
        \toprule[1.5pt]
        \multirow{2}{*}{\textbf{Module}} & \multirow{2}{*}{\textbf{Voltage [V]}}    & \multicolumn{2}{c}{\textbf{Current [mA]}} & \multicolumn{2}{c}{\textbf{Power [mW]}} \\
                                         &                                          & \textbf{Min.} & \textbf{Max.}             & \textbf{Min.} & \textbf{Max.} \\
        \midrule
        OBDH                & 3,3   & 35    & 200   & 115   & 660 \\
        TTC ($\mu$C)        & 3,3   & 40    & 40    & 132   & 132 \\
        TTC (radio module)  & 5     & 10    & 650   & 33    & 3250 \\
        EPS (digital part)  & 7,4   & 50    & 260   & 165   & 858 \\
        EPS (heater)        & 7,4   & 675   & 675   & 5000  & 5000 \\
        Antenna module      & 3,3   & 60    & 550   & 200   & 1800 \\
        Payload EDC         & 5     & 250   & 250   & 1250  & 1250 \\
        \bottomrule[1.5pt]
    \end{tabular}
    \caption{Power requirements of the subsystems and payloads of the satellite.}
    \label{tab:power-requirements}
\end{table}

Using the information presented in \autoref{tab:power-requirements}, and the activation periods defined for each module (the duty cycle), we arrive at the average satellite consumption present in \autoref{tab:power-consumption}.

\begin{table}[!h]
    \centering
    \begin{tabular}{lccccc}
        \toprule[1.5pt]
        \textbf{Module} & \textbf{Duty Cycle [\%]}    & \textbf{Power [mW]} \\
        \midrule
        OBDH                    & 100   & 115 \\
        TTC (radio 1 RX)        & 95    & 65 \\
        TTC (radio 1 TX)        & 5     & 3250 \\
        TTC (radio 2 RX)        & 95    & 65 \\
        TTC (radio 2 TX)        & 5     & 3250 \\
        EPS                     & 100   & 320 \\
        BAT (idle)              & 90    & 0 \\
        BAT (heater full)       & 10    & 5000 \\
        Antenna (deployment)    & 0     & 1800 \\
        Antenna (deployed)      & 100   & 35 \\
        Payload EDC             & 100   & 1250 \\
        Payload Harsh           & 0     & 330 \\
        Payload-X               & 0     & 1000 \\
        \cmidrule{2-3}
        Satellite               & \multicolumn{2}{c}{$\cong$ 2668} \\
        \bottomrule[1.5pt]
    \end{tabular}
    \caption{Power consumption of the subsystems and payloads of the satellite.}
    \label{tab:power-consumption}
\end{table}

The duty cycles of \autoref{tab:power-consumption} were defined according to the following assumptions:

\begin{itemize}
    \item One of the EDC payload is always off (cold redundancy).
    \item The Payload-X and the Harsh payload are turned on just during limited periods and only with telecommands.
\end{itemize}

As can be seen from \autoref{fig:sp_sim_power} and \autoref{tab:power-consumption}, there is a slight positive margin of \textbf{76,9 mW} in the power budget.

\subsection{Orbit analysis}

To define the orbit parameters and simulate the behaviour of the satellite during its operation, the GMAT software was used \cite{gmat}. The orbit parameters was based on the FloripaSat-I TLE, but with a lower altitude. These parameters can be seen in \autoref{tab:orbit-parameters}.

\begin{table}[!h]
    \centering
    \begin{tabular}{lcc}
        \toprule[1.5pt]
        \textbf{Parameters} & \textbf{Value} & \textbf{Unit} \\
        \midrule
        Altitude                & 550           & km \\
        Eccentricity            & 0,0015051     & $^{\circ}$ \\
        Inclination             & 97,9750       & $^{\circ}$ \\
        RAAN                    & 85,5100       & $^{\circ}$ \\
        Arg. of Perigee (AOP)   & 194,87        & $^{\circ}$ \\
        TA                      & 99,8877       & $^{\circ}$ \\
        \bottomrule[1.5pt]
    \end{tabular}
    \caption{Initial orbit parameters (adapted from FloripaSat-I).}
    \label{tab:orbit-parameters}
\end{table}

The parameters of the simulation on GMAT was based on \cite{marino2016} and can be seen below:

\begin{itemize}
    \item Force model for gravitational field: ``\textit{Earth Gravitational Model 1996 (EGM96)}''
    \item Propagator: ``\textit{PrinceDorman78}''
    \item Drag coefficient: 2,2
    \item Drag atmosphere model: ``\textit{Mass Spectrometry and Incoherent Scatter (MSISE90)}''
    \item Epoch: 01 Jan 2022 11:59:28.000
\end{itemize}

The \autoref{fig:fsat2-gmat} shows the 3D representation of the FloripaSat-2 orbit simulation, \autoref{fig:fsat2-gmat-groundtrack} shows the ground track of the first day of operation.

\begin{figure}[!ht]
    \begin{center}
        \includegraphics[width=0.6\columnwidth]{figures/fsat2-gmat.png}
        \caption{FloripaSat-2 orbit simulation on GMAT.}
        \label{fig:fsat2-gmat}
    \end{center}
\end{figure}

\begin{figure}[!ht]
    \begin{center}
        \includegraphics[width=\columnwidth]{figures/fsat2-gmat-groundtrack.pdf}
        \caption{FloripaSat-2 simulated groundtrack.}
        \label{fig:fsat2-gmat-groundtrack}
    \end{center}
\end{figure}

The next sections present some analysis based on the results obtained on the simulations executed on GMAT.

%The source files of the GMAT simulation are available in \cite{fsat2-mechanical}.

\subsubsection{Lifetime Analysis}

Considering the same parameters of FloripaSat-I, but with an initial altitude of 550 km, the simulations on GMAT showed that the satellite decays approximately in 2000 days ($\cong$ 5 years), as can be seen in \autoref{fig:lifetime-analysis}.

\begin{figure}[!ht]
    \begin{center}
        \includegraphics[width=\columnwidth]{curves/lifetime.pdf}
        \caption{Lifetime analysis on GMAT.}
        \label{fig:lifetime-analysis}
    \end{center}
\end{figure}
