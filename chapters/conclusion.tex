%
% conclusion.tex
%
% Copyright (C) 2022 by Universidade Federal de Santa Catarina.
%
% GNSS Networks Based on Small Satellites
%
% This work is licensed under the Creative Commons Attribution-ShareAlike 4.0
% International License. To view a copy of this license,
% visit http://creativecommons.org/licenses/by-sa/4.0/.
%

%
% \brief Conclusion chapter.
%
% \author Gabriel Mariano Marcelino <gabriel.mm8@gmail.com>
%
% \version 0.0.0
%
% \date 2019/11/30
%

\chapter{Next steps} \label{ch:conclusion}

%Como conclusão desta etapa preliminar, pela análise efetuada até o momento, a implantação do sistema proposto demonstra-se possível.

%Como próximas etapas, pretende-se refinar as análises efetuadas até então. Além de adicionar uma análise de órbita, onde se chegará a possíveis quantidades de objetos necessários para diferentes tipos de órbita.

%Também pretense-se elaborar o projeto do payload proposto, para uma possível futura implementação e implantação em umas das futuras missões do grupo de pesquisa. Para um teste efetivo do funcionamento do sistema proposto e do payload desenvolvido, uma missão composta por múltiplos satélites também será apresentada e proposta.

As a conclusion of this preliminary stage, based on the analysis carried out so far, the implementation of the proposed system appears to be feasible.

As next steps, we intend to refine the analyses conducted so far. In addition, an orbit analysis will be added to determine the number of objects required for different types of orbits. As well as the minimum number of satellites required to achieve a certain accuracy considering a receiver being operated by a user of the system.

We also plan to develop the design of the proposed payload for a possible future implementation and deployment in one of the research group's future missions. To effectively test the proposed system and developed payload, a mission composed of multiple satellites will also be presented and proposed.