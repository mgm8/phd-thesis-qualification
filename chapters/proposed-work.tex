%
% proposed-work.tex
%
% Copyright (C) 2022 by Universidade Federal de Santa Catarina.
%
% GNSS Networks Based on Small Satellites
%
% This work is licensed under the Creative Commons Attribution-ShareAlike 4.0
% International License. To view a copy of this license,
% visit http://creativecommons.org/licenses/by-sa/4.0/.
%

%
% \brief Proposed work chapter.
%
% \author Gabriel Mariano Marcelino <gabriel.mm8@gmail.com>
%
% \version 0.1.0
%
% \date 2021/06/14
%


\chapter{Proposed Work} \label{ch:proposed-work}

Neste capítulo apresenta-se o trabalho proposto por essa tese. Que resumidamente consiste no estudo da viabilidade da implantação de redes de GNSS utilizando satélites de pequeno porte, podendo operar tanto em órbitas convencionais para este tipo de aplicação, mas principalmente em órbitas baixas, em altitudes que vêm sendo mais exploradas atualmente por esses tipos de dispositivos.

\section{Orbit study}

.

\section{Communication study}

.

\section{Possible solutions}

\begin{itemize}
    \item Adicionar transmissão de sinais de GNSS às mega-constelações atuais, precisando somente adicionar relógio de alta precisão a cada satélite. Considerando custo e praticidade, essa seria possivelmente a melhor alternativa.
    \item Montar uma rede dedicada, com satélites feitos exclusivamente para emissão de sinais de GNSS. Essa seria solução de maior custo e provavelmente a mais ineficiente em termos de praticidade.
\end{itemize}
